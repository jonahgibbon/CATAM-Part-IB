\documentclass[10pt,a4paper,notitlepage]{article}
\usepackage[utf8]{inputenc}
\usepackage{amsmath}
\usepackage{amsfonts}
\usepackage{amssymb}
\usepackage{fullpage}
\usepackage{multirow}
\usepackage{fancyhdr}
\usepackage{float}

\usepackage[backend=bibtex,style=verbose-trad2]{biblatex}
\author{Jonah Gibbon}
\bibliography{References}

\usepackage{graphicx}
\graphicspath{{./Graphics/}}

\pagestyle{fancy}
\fancyhf{}
\renewcommand{\headrulewidth}{0pt}
\cfoot{Page \thepage\ of \pageref{LastPage}}

\newcommand{\abs}[1]{\lvert#1\rvert}
\newcommand{\Z}{\mathbb{Z}}
\newcommand{\Q}{\mathbb{Q}}
\newcommand{\C}{\mathbb{C}}
\newcommand{\N}{\mathbb{N}}
\newcommand{\R}{\mathbb{R}}
\newcommand{\Rank}{\text{Rank}}
\newcommand{\Ker}{\text{Ker}}
\newcommand{\Span}{\text{Span}}
\newcommand{\Nul}{\text{Nul}}

\usepackage{nameref}


\begin{document}
\begin{large}
\section*{\centering \large Question 1}
\end{large}
Below is test data from \nameref{subsec:Code 1.1} referenced on page \pageref{subsec:Code 1.1} using $p=11 $.
\begin{table}[H]
\begin{center}
\begin{tabular}{|c|cccccccccc|}
\hline 
$a$ & 1 & 2 & 3 & 4 & 5 & 6 & 7 & 8 & 9 & 10 \\ 
\hline 
$a^{-1}$ & 1 & 6 & 4 & 3 & 9 & 2 & 8 & 7 & 5 & 10 \\ 
\hline 
\end{tabular} 
\end{center}
\caption{Test data for Code 1.1}
\end{table}
To improve this procedure, we exploit the fact that each $2\leq a \leq p-2$ has a unique distinct inverse. Therefore by storing \texttt{inverses(a)=b} and \texttt{inverses(b)=a} in one iteration, and assuming that each of the inverses are randomly distributed in $\Z_{p}$, there would be $(p+1)/2$ iterations of the code.  For large $p$ this would speed up the procedure by a factor of 2, however I appreciate the code would check if each entry had already been calculated, so this factor would be slightly less than 2. The modification is referenced as \nameref{subsec:Code 1.2} on page \pageref{subsec:Code 1.2}. I ran both codes with $p=7121$; the first took 0.92 seconds, the second took 0.58 seconds, confirming this idea.\\
\section*{\centering \large Question 2}
Suppose $a_{i}$ was the inverse for $i$, so $i\cdot a_{i}\equiv 1 \mod p$. There are 3 operations to check if mod($a_{i}$*$b$)==1 for some $b$, and this will iterate $i$ times when finding the inverse for $a_{i}$. So there will be approximately 
\begin{equation}
\sum_{i=1}^{p-1}\left(3i\right)=\frac{3}{2}p(p-1)
\end{equation}
operations. Creating a zero vector at the start uses $p-1$ operations, so the complexity of the program is $O(p^{2})$.\\
\section*{\centering \large Question 3}
The results in the second column of Table \ref{Echelon} were produced by \nameref{subsec:Code 3.1}, referenced on page \pageref{subsec:Code 3.1}.\footnote{Code 1.1 was used at the start of Code 3.1 to produce the \texttt{inverses} vector, it has not been included in the referencing.}


\begin{table}[H]
\centering
\begin{tabular}{|c|c|c|c|}
\hline
Matrix & Echelon Form & Rank & Basis for Row Space\\
\hline
$A_{1}$ mod(3) & $\begin{pmatrix}
1 & 0 & 1 & 1 & 2\\0 & 1 & 1 & 2 & 1\\0 & 0 & 1 & 2 & 0\\0 & 0 & 0 & 0 &0
\end{pmatrix}$ & 3 & $\left\lbrace \begin{array}{c}
\begin{pmatrix}1&0&1&1&2\end{pmatrix},\\
\begin{pmatrix}0&1&1&2&1\end{pmatrix},\\
\begin{pmatrix}0&0&1&2&0\end{pmatrix}.\\
\end{array} \right\rbrace$\\
\hline
$A_{1}$ mod(29) & $\begin{pmatrix}
1 & 0 & 22 & 26 & 11\\ 0 & 1 & 7 & 2 & 10\\ 0 & 0 & 1 & 1 & 26\\ 0 & 0 & 0 & 1 & 1
\end{pmatrix}$ & 4 & $\left\lbrace \begin{array}{c}
\begin{pmatrix}1&0&22&26&11\end{pmatrix},\\
\begin{pmatrix}0&1&7&2&10\end{pmatrix},\\
\begin{pmatrix}0&0&1&1&26\end{pmatrix},\\
\begin{pmatrix}0&0&0&1&1\end{pmatrix}.\\
\end{array} \right\rbrace$\\
\hline
$A_{2}$ mod(23) & $\begin{pmatrix}
1 & 18 & 21 & 10 & 4 & 16\\ 0 & 1 & 4 & 0 & 15 & 10\\ 0 & 0 & 1 & 9 & 14 & 7\\ 0 & 0 & 0 & 0 & 0 & 0 
\end{pmatrix}$ & 3 & $\left\lbrace \begin{array}{c}
\begin{pmatrix}1&18&21&10&4&16\end{pmatrix},\\
\begin{pmatrix}0&1&4&0&15&10\end{pmatrix},\\
\begin{pmatrix}0&0&1&9&14&7\end{pmatrix}.\\
\end{array} \right\rbrace$\\
\hline
\end{tabular}
\caption{Matrices converted into echelon form using Gaussian elimination}
\label{Echelon}
\end{table}
The rows of a matrix in echelon form are linearly independent. Let $r_{i}$ be the $i$-th row of the matrix. Suppose row $r_{i}$ was a linear combination of other rows. By induction on $j$, where $j<i$, it cannot contain a combination of rows $r_{j}$ since $r_{j}(l(j))=1$, whereas $r_{i}(l(j)=0$. So it must only be a combination of rows $r_{k}$ where $k>i$. However this is not the case, since $r_{i}(l(i))=1$ and $r_{k}(l(i))=0$ for all $k$. So $r_{i}$ cannot be written as a combination of the other rows and therefore the rows must be linearly independent. It follows that the rows of a matrix in echelon form is a basis for its own row-space. Using the fact $\Rank(A)=\Rank(A^{T})$ we deduce that $\Rank(A)$ is the value of $r$. \par

 

\section*{\centering \large Question 4}
Suppose 
\begin{equation}
x_{l(r)}+\sum_{j=l(r)+1}^{n}a_{r,j}\cdot x_{j}=0
\end{equation}
where $A=(a_{i,j})$. There exist $n-l(r)$ vectors $v$ where $v_{j}=1$ for some $j>l(r)$, $v_{l(r)}=-a_{r,j}$ and zero everywhere else. Let $S$ be the set of these vectors. We now iterate from $i=r-1$ up to $i=1$, each time doing the following:\\
1. Create $l(i)-l(i+1)-1$ vectors of the form $v_{j}=1$ for some $l(i-1)<j<l(i)$ and 0 everywhere else, and add them to $S$.\\
2. For all vectors $v\in S$, set
\begin{equation}
v_{l(i)}=-\sum_{j=l(i)+1}^{n}a_{i,j}v_{j}
\end{equation}
Once this process finishes, we will have $n-l(r)+\sum_{i=1}^{r-1}(l(i+1)-l(i)-1)=n+1-r-l(1)=n-r$ vectors in $S$.\footnote{This is only the case when $l(1)=1$. However we may assume this true, since the matrices in this project don't have a column of 0's at their start. If, however,  $A$ had $k$ 0 columns on the front of it, we could introduce $k$ more kernel vectors of the form $v_{i}=1$ where $i\leq k$ and zero everywhere else. We would combine these vectors with the vectors produced by the iteration in Question 4, however these vectors would now have $k$ zero entries on the top of them.} The vectors in $S$ have been constructed so that they are linearly independent, since for each $j\neq l(i)$ for some $i$, there is a corresponding vector $v$ with $v_{j}=1$. All other vectors in $S$ have their $j$th entry equal to 0, so each vector cannot be written as a linear combination of the others. By the rank-nullity theorem, we have a complete basis for the kernel of $A$. This is how \nameref{subsec:Code 4.1} referenced on page \pageref{subsec:Code 4.1} has computed the kernels of the matrices in Table \ref{Kernel}.\footnote{Code 1.1 was used at the start of Code 4.1 but has not been referenced. The function \texttt{Echelon}, as defined in Code 3.1, has also been used but also not included.}

\begin{table}[H]
\centering
\begin{tabular}{|c|c|c|}
\hline
Matrix & Echelon Form & Basis for Kernel\\
\hline
$B_{1}$ mod(2) & $\begin{pmatrix}
1&0&1&0&1&0\\0&1&0&1&1&0\\0&0&1&0&1&1\\0&0&0&1&1&1\\0&0&0&0&0&0\\\end{pmatrix}$&
$\begin{pmatrix}
0\\0\\1\\1\\1\\0\end{pmatrix},\begin{pmatrix}1\\1\\1\\1\\0\\1\end{pmatrix}$\\
\hline
$B_{1}$ mod(11) & $\begin{pmatrix} 1 & 7 & 4 & 6 & 9 & 3\\ 0 & 1 & 3 & 4 & 0 & 2\\ 0 & 0 & 1 & 4 & 10 & 10\\ 0 & 0 & 0 & 0 & 1 & 3\\ 0 & 0 & 0 & 0 & 0 & 1 \end{pmatrix}$&$\begin{pmatrix}
9\\8\\7\\1\\0\\0\end{pmatrix}$\\
\hline
$B_{1}$ mod(17) & $\begin{pmatrix} 1 & 10 & 14 & 9 & 5 & 13\\ 0 & 1 & 1 & 6 & 10 & 3\\ 0 & 0 & 1 & 6 & 3 & 10\\ 0 & 0 & 0 & 1 & 2 & 1\\ 0 & 0 & 0 & 0 & 1 & 16 \end{pmatrix}$&$\begin{pmatrix}
7\\0\\5\\14\\1\\1\end{pmatrix}$\\
\hline
$B_{2}$ mod(23) & $\begin{pmatrix} 1 & 10 & 14 & 1 & 3 & 2\\ 0 & 1 & 20 & 3 & 16 & 11\\ 0 & 0 & 1 & 17 & 7 & 5\\ 0 & 0 & 0 & 1 & 20 & 18\\ 0 & 0 & 0 & 0 & 1 & 14 \end{pmatrix}$&$\begin{pmatrix}
6\\6\\9\\9\\9\\1\end{pmatrix}$\\
\hline
\end{tabular}
\caption{Kernels produced by Code 4.1}
\label{Kernel}
\end{table}
\section*{\centering \large Question 5}
If $U\subseteq F^{n}$ then
\begin{equation}
\begin{aligned}
\dim(U)+\dim(U^{\circ})=n \\
\end{aligned}
\end{equation}
Note that when $U$ is the row space for a matrix $A$, we recover the rank-nullity theorem; $\Rank(A)+\Nul(A)=n$
\section*{\centering \large Question 6}
Inputting $A_{1}$ into \nameref{subsec:Code 4.1} produced the middle two columns of Table \ref{TabQ6}. The last column was produced by inputting $(U^{\circ})^{T}$ into \nameref{subsec:Code 4.1} again.
\begin{table}[H]
\centering
\begin{tabular}{|c|c|c|c|}
\hline
$U=$ Row space & Echelon Form & $U^{\circ}$ & $(U^{\circ})^{\circ}$\\
\hline
$A_{1}$ & $\begin{pmatrix}1&0&5&3&12\\0&1&7&2&10\\0&0&1&4&7\\0&0&0&1&1\end{pmatrix}$ & $\begin{pmatrix}6\\13\\16\\18\\1\end{pmatrix}$ & $\begin{pmatrix}1&1&0&0&0\\0&1&17&0&0\\0&0&1&16&0\\0&0&0&1&1\end{pmatrix}$\\
\hline
\end{tabular}
\caption{Kernels produced by Code 4.1}
\label{TabQ6}
\end{table}
The following operations in this order converts $(U^{\circ})^{\circ}$ back to $U$: $S(1,1,2), S(2,10,3), S(2,9,4), S(3,12,4)$. This verifies that $(U^{\circ})^{\circ}=U$.

\section*{\centering \large Question 7}
I will prove that $\dim(U) +\dim(W) =\dim(U+W) +\dim(U\cap W)$. Suppose $\dim(U\cap W)=q$ with basis $\{x_{i}\}$ where $1\leq i \leq q$, and that $\dim(U)=a$ and $\dim(W)=b$. We may extend the basis $\{x_{i}\}$ so that the basis of $U$ is $\{ x_{i},u_{j}\}$ where $q+1\leq j\leq a$ and that the basis for $W$ is $\{ x_{i}, w_{k}\}$ where $q+1\leq k \leq b$. Therefore $\{x_{i},  u_{j}, w_{k}\}$ spans $U+W$. Assume that 
\begin{equation}
\sum_{i=1}^{q}\lambda_{i}x_{i} +\sum_{j=q+1}^{a}\mu_{j}u_{j}+\sum_{k=q+1}^{b}\eta_{k}w_{k}=0
\end{equation}
for constants $\lambda_{i},\mu_{j}, \eta_{k}$. If at least one $\eta_{k}\neq 0$ then there is a linear combination of vectors $w_{k}$ that is in $U\cap V$, a contradiction that $\{x_{i}, w_{k}\}$ was a basis. Therefore all $\eta_{k}=0$. The fact that $\{x_{i}, u_{j}\}$ is a basis then implies $\lambda_{i}=\mu_{j}=0$ for all $i,j$. We conclude that $\{x_{i}, u_{j}, w_{k}\}$ is a set of linearly independent vectors and so forms a basis for $U+W$. There are $a+b-q$ vectors in this basis and so $\dim(U+V)+\dim(U\cap V)=a+b=\dim(U)+\dim(V)$.\\

The following is an explanation of how the program works. \nameref{subsec:Code 3.1} already calculates a basis for $U$ and $W$ from matrices $A$ and $B$, and so this is used at the start. If $U$ had basis $\{u_{i}\}$ and $W$ had basis $\{w_{j}\}$, then $S=\Span\{u_{i},w_{j}\}=U+W$. 
This is because if $x\in U+W,  x=u+w$ for some $u\in U,w\in W$ and so $U+W\subseteq S$.  Equally if $x\in S$, $x=\sum_{i} \lambda_{i}u_{i}+\sum_{j}\mu_{j}v_{j}$, for constants $\lambda_{i},\mu_{j}$, which lies in $U+W$, so $S\subseteq U+W$, and therefore $S=U+W$. Inputting the matrix 
$\begin{pmatrix}
A\\B
\end{pmatrix}$
into Code 3.1 will give a linearly independent set of vectors with the same span, i.e. a basis for $U+W$. 
To calculate $U\cap W$, we must first find a basis for $U^{\circ}+W^{\circ}$, by using Code 4.1 to find the annihilators and then inputting the matrix $(\ker(A)\hspace{0.5em}\ker(B))^{T}$ into Code 3.1. Then by using Code 4.1 again we may find $(U^{\circ}+W^{\circ})^{\circ}=U\cap W$.\\



\nameref{subsec:Code 7.1} described above and referenced on page \pageref{subsec:Code 7.1} was used to produce the results in Tables \ref{Q7.1}-\ref{Q7.2}.\footnote{Code 1.1 was used at the start of Code 7.1, but has not been included in the referencing. The functions \texttt{Echelon} and \texttt{Ker}, defined in the previous codes, were also used and have also not been referenced.}

\begin{table}[H]
\centering
\begin{tabular}{|c|c|}
\hline
Basis for $U$ & Basis for $W$ \\
\hline
$\left\lbrace\begin{array}{cccccc}
(1&10&0&6&2&8)\\
(0&1&9&5&8&2)\\
(0&0&1&5&6&0)\\
(0&0&0&1&0&9)
\end{array}
\right\rbrace$ & $\left\lbrace\begin{array}{cccccc}
(1     &7     &4     &6     &9     &3)\\
 (    0     &1     &3     &4     &0     &2)\\
  (   0     &0     &1    & 4   & 10   & 10)\\
   (  0   &  0   &  0  &   0 &    1   &  3)\\
    ( 0   &  0 &    0   &  0  &   0   &  1)
\end{array}
\right\rbrace$ \\
\hline Basis for $U+W$ & Basis for $U\cap W$\\
\hline $\left\lbrace\begin{array}{cccccc}
(1    &10   &  0   &  6   &  2  &   8)\\
 (    0    & 1  &   9   &  5   &  8   &  2)\\
  (   0  &   0   &  1   & 5    & 6   &  0)\\
  (   0  &   0    & 0  &   1  &   0  &   9)\\
    ( 0   &  0    & 0  &   0   &  1  &   4)\\
    (0   &  0     &0  &   0   &  0     &1)\\
\end{array}
\right\rbrace$ & $\left\lbrace\begin{array}{cccccc}
 (1    & 7     &6     &3     &0     &0)\\
  (   0  &   1   &  9   &  6   &  8   &  0)\\
   (  0     &0     &1     &4     &6     &2)\\
\end{array}
\right\rbrace$ \\
\hline
\end{tabular}
\caption{$A=A_{2}$ and $B=B_{1}$ working modulo 11}
\label{Q7.1}
\end{table}

\begin{table}[H]
\centering
\begin{tabular}{|c|c|}
\hline
Basis for $U$ & Basis for $W$ \\
\hline
$\left\lbrace\begin{array}{ccccccc}
(1 & 0 & 0 & 0 & 3 & 0 & 0)\\
(0 & 1 & 0 & 4 & 5 & 12 & 0)\\
(0 & 0 & 1 & 0 & 8 & 0 & 0)\\
(0 & 0 & 0 & 1 & 4 & 11 & 13)\\
(0 & 0 & 0 & 0 & 1 & 17 & 6)
\end{array}
\right\rbrace$ & $\left\lbrace\begin{array}{ccccccc}
(1	& 10	& 9	& 0	& 6	& 3	& 0)\\
(0	& 1	& 0	& 2	& 0	& 4	& 14)
\end{array}
\right\rbrace$ \\
\hline Basis for $U+W$ & Basis for $U\cap W$\\
\hline $\left\lbrace\begin{array}{ccccccc}
(1	& 0	& 0	& 0	& 3	& 0	& 0)\\
(0	& 1	& 0	& 4	& 5	& 12	& 0)\\
(0	& 0	& 1	& 0	& 8	& 0	& 0)\\
(0	& 0	& 0	& 1	& 4	& 11	& 13)\\
(0	& 0	& 0	& 0	& 1	& 17	& 6)\\
(0	& 0	& 0	& 0	& 0	& 1	& 14)\\
(0	& 0	& 0	& 0	& 0	& 0	& 1)\\
\end{array}
\right\rbrace$ & $\left\lbrace\begin{array}{ccccccc}
(0 & 0 & 0 & 0 & 0 & 0 & 0)
\end{array}
\right\rbrace$ \\
\hline
\end{tabular}
\caption{$A=A_{3}$ and $B=\Ker(A_{3})$ working modulo 19}
\end{table}
\begin{table}[H]
\centering
\begin{tabular}{|c|c|}
\hline
Basis for $U$ & Basis for $W$ \\
\hline
$\left\lbrace\begin{array}{ccccccc}
(1	& 0	& 0	& 0	& 3	& 0	& 0)\\
(0 & 1	& 0	& 3	& 4	& 2	& 0)\\
(0	& 0	& 1	& 0	& 6	& 0	& 0)\\
(0	& 0	& 0	& 1	& 3	& 2	& 4)\\
(0 & 0	& 0	& 0	& 1	& 0	& 0)
\end{array}
\right\rbrace$ & $\left\lbrace\begin{array}{ccccccc}
(0	& 1	& 0	& 3	& 0	& 2	& 0)\\
(0	& 0	& 0	& 1	& 0	& 2	& 4)\\
\end{array}
\right\rbrace$ \\
\hline Basis for $U+W$ & Basis for $U\cap W$\\
\hline $\left\lbrace\begin{array}{ccccccc}
(1	& 0	& 0	& 0	& 3	& 0	& 0)\\
(0	& 1	& 0	& 3	& 4	& 2	& 0)\\
(0	& 0	& 1	& 0	& 6	& 0	& 0)\\
(0	& 0	& 0	& 1	& 3	& 2	& 4)\\
(0	& 0	& 0	& 0	& 1	& 0	& 0)
\end{array}
\right\rbrace$ & $\left\lbrace\begin{array}{ccccccc}
(0	& 1	& 0	& 3	& 0	& 2	& 0)\\
(0	& 0	& 0	& 1	& 0	& 2	& 4)
\end{array}
\right\rbrace$ \\
\hline
\end{tabular}
\caption{$A=A_{3}$ and $B=\Ker(A_{3})$ working modulo 7}
\label{Q7.2}
\end{table}

\section*{\centering \large Question 8}
In the last part of Question 7 the intersection of the kernel and the row space is non trivial.  If we were working over the real numbers, this would be impossible. Suppose that a non zero vector $x$ lay in the kernel and row space of $A$, then $Ax^{T}=0$. Therefore the dot product of $x^{T}$ with any of the rows of $A$ would equal zero. There also exists a linear combination of rows that equals $x$, so we conclude that $xx^{T}=0$. Working over $\R$, this implies that $x$ is the zero vector, and so the intersection must be trivial. However we cannot conclude that $x=0$ if we work over GF($p$). To give an example, take $p=5$ and $x=(2,4)$.
\pagebreak
\section*{\centering Code}
\subsection*{\centering\normalsize Code 1.1} \label{subsec:Code 1.1}
\begin{verbatim}
%Initial data
p=11;
a=1;
b=1;

tic
inverses=zeros(p-1,1);
while a<p
    while 1
        %This checks if b is the inverse of a, and if so, stores it then
        %breaks the loop.
        if mod(a*b,p)==1
            inverses(a)=b;
            b=1;
            break
        end
        b=b+1;
    end
    a=a+1;
end
disp(inverses)
toc
\end{verbatim}
\pagebreak
\subsection*{\centering\normalsize Code 1.2}\label{subsec:Code 1.2}
\begin{verbatim}
%Initial data
p=11;
a=1;
b=1;
tic
inverses=zeros(p-1,1);
while a<p
    if inverses(a)==0
        while 1
            %This checks if b is the inverse of a, and if so, stores it then
            %breaks the loop.
            if mod(a*b,p)==1
                inverses(a)=b;
                inverses(b)=a;
                b=1;
                break
            end
            b=b+1;
        end
    end
    a=a+1;
end
disp(inverses)
toc
\end{verbatim}
\pagebreak
\subsection*{\centering\normalsize Code 3.1} \label{subsec:Code 3.1}
\begin{verbatim}
%Initial data
m=[0,1,7,2,10;8,0,2,5,1;2,1,2,5,5;7,4,5,3,0];
p=3;

[m,rank]=Echelon(m,p,inverses)

function [m,rank] = Echelon(m,p,inverses)
%First this makes sure all elements of the matrix lie within mod p
m=mod(m,p);
%The dimensions of the matrix are used for the limits of the while loop
dimensions=size(m);
columns=1;
rows=1;
%rank tracks how many rows have already changed to echelon form
rank=0;
while columns<=dimensions(2)
    %For each column, we search downwards until we find a non-zero element.
    %If we are successful, we convert this column and the top most possible
    %row into echelon form, before converting the 'lower-right' matrix into
    %echelon form. Otherwise we move onto the next column.
    while rows<=dimensions(1)
        if m(rows,columns)~=0
            m=T(m,rows,rank+1);
            m=D(m,rank+1,m(rank+1,columns),inverses,p);
            rows=rank+2;
            while rows<=dimensions(1)
                m=S(m,rows,m(rows,columns),rank+1,p);
                rows=rows+1;
            end
            rank=rank+1;
            break
        end
        rows=rows+1;
    end
    %We continue to convert the 'lower-right' matrix into echelon form
    rows=rank+1;
    columns=columns+1;
end
end

function m = T(m,i,j)
    v=m(i,:);
    m(i,:)=m(j,:);
    m(j,:)=v;
end

%This divides a row i by a by indexing its inverse from 'inverses'
function m = D(m,i,a,inverses,p)
    v=m(i,:);
    v=mod(inverses(a)*v,p);
    m(i,:)=v;
end

%This takes a multiple a of row j and minuses this from row i
function m = S(m,i,a,j,p)
    if i==j
        error("i should not equal j")
    else
        m(i,:)=m(i,:)-a*m(j,:);
    end
    %This makes sure that the matrix still lies in mod(p)
    m=mod(m,p);
end
\end{verbatim}
\pagebreak
\subsection*{\centering\normalsize{Code 4.1}}\label{subsec:Code 4.1}
\begin{verbatim}
m=[4,6,5,2,3,1;5,0,3,0,1,0;1,5,7,1,0,12;5,5,0,3,1,7;2,1,2,4,0,5];
p=2;

[m,rank]=Echelon(m,p,inverses)
ker=Ker(m,p,rank)

function basis = Ker(m,p,rank)
%This creates a blank array to be filled with vectors for the basis
basis=zeros(size(m,2),size(m,2)-rank);

%This resets the rows,columns and noBasis (which represents the number of
%basis vectors created) counters.
rows=size(m,1);
columns=1;
noBasis=1;
lastColumn=size(m,2)+1;

%This iterates through until it finds a non-zero element
while rows >= 1
    while columns <= size(m,2)
        if m(rows,columns)~=0
            %If there is a 1 in the far right of the matrix, this ensures
            %the last element of the vector is 0
            if columns==size(m,2)
                lastColumn=columns;
            else
                %This iterates between the column where the non-zero
                %element is, and the last column that caused a vector to be
                %produced. It creates new linearly-independant vectors for
                %the basis, before ensuring that the vector lies in the
                %kernal
                thisColumn=columns;
                columns=columns+1;
                while columns<lastColumn
                    basis(columns,noBasis)=1;
                    noBasis=noBasis+1;
                    columns=columns+1;
                end
                basis(thisColumn,:)=-m(rows,:)*basis;
                lastColumn=thisColumn;
            end
            break
        end
        columns=columns+1;
    end
    columns=1;
    rows=rows-1;
end
%This ensures that the vector lies within mod(p)
basis=mod(basis,p);
end
\end{verbatim}
\pagebreak
\subsection*{\centering\normalsize{Code 7.1}}\label{subsec:Code 7.1}
\begin{verbatim}
A=[6,16,11,14,1,4;7,9,1,1,21,0;8,2,9,12,17,7;2,19,2,19,7,12];
B=[4,6,5,2,3,1;5,0,3,0,1,0;1,5,7,1,0,12;5,5,0,3,1,7;2,1,2,4,0,5];
p=11;

%This calculates the basis for U and W, as well as the kernels of A and B
[U,rank]=Echelon(A,p,inverses);
disp('U')
disp(U)
KerU=Ker(U,p,rank);
[W,rank]=Echelon(B,p,inverses);
disp('W')
disp(W)
KerW=Ker(W,p,rank);

%This calculates the basis for U+W, using the theory described in Q7
V=[U;W];
V=Echelon(V,p,inverses);
disp('U+W')
disp(V)

%This calculates a basis for U^o+W^o, and then calculates the kernel of
%this for the basis of U intersect W.
X=[KerU KerW]';
[X,rank]=Echelon(X,p,inverses);
X=Ker(X,p,rank)';
X=Echelon(X,p,inverses);
disp('U intersect W')
disp(X)
\end{verbatim}
\label{LastPage}
\end{document}
